% !TEX encoding = IsoLatin
\chapter*{Introduction}
\addcontentsline{toc}{chapter}{Introduction}

% introduction	
\paragraph{SUJET}
\paragraph{}Dans ce bureau d'�tude nous consid�rons le syst�me d'une turbine � gaz. Ce type de syst�me est choisi afin d'appliquer une d�marche industrielle de conception. 

\paragraph{ETAPES}
\paragraph{}L'objectif est de concevoir une commande pour la turbine qui va respecter un cahier des charges. Plusieurs comp�tences th�oriques sont n�cessaires pour la r�solution du probl�me pos�. Dans un premier temps, nous mod�liserons le syst�me avec un mod�le math�matique d�crivant son comportement. Ensuite, nous allons lin�ariser autour d'un point de fonctionnement et choisir les diff�rentes solutions possibles pour la strat�gie de contr�le afin de mieux satisfaire les sp�cifications. Nous expliquerons l'utilit� d'un filtre de Kalman pour mieux estimer le mod�le. Une commande par retour d'�tat et une discussion introductive aux techniques de commande robuste seront enfin pr�sent�es.

\paragraph{SUPPORT}
	         
\paragraph{}\textit{L'ensemble des sources du projet (Matlab/Simulink) est disponible � l'adresse suivante : \url{https://github.com/mape1991/BEcommandeAA}} 