% !TEX encoding = IsoLatin
\chapter*{Introduction}
\addcontentsline{toc}{chapter}{Introduction}

% introduction	
\paragraph{SUJET}
\paragraph{}Dans ce bureau d'�tudes nous consid�rons le syst�me d'une turbine � gaz. Ce type de syst�me est choisi afin d'appliquer une d�marche industrielle de conception. 

\paragraph{ETAPES}
\paragraph{}L'objectif est de concevoir une commande pour la turbine qui va respecter un cahier des charges. On va appeler plusieurs comp�tences th�oriques dans ce rapport. Dans un premier temps on mod�lise le syst�me avec un mod�le math�matique qui d�crit son comportement. Ensuite on va lineariser autour d'un point de fonctionnement et choisir les diff�rentes solutions possibles pour la commande afin de mieux satisfaire les sp�cifications. On a d�velopp� un observateur de Kalman pour mieux estim� le mod�le. Nous avons mis en point une commande par retour d'�tat et une commande robuste de fa�on � r�duire les d�lais et le co�t de d�veloppement. A la fin on a pr�sent� une validation des objectifs.

\paragraph{SUPPORT}
	         
\paragraph{\textit{Note :}}
\paragraph{}\textit{Le script Matlab r�alis� pour ce projet est dans l'annexe.} 