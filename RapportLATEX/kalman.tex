% !TEX encoding = IsoLatin
% --------------------------------------------------------------------------------------------------------------------------------------------------------------------------- %
%\begin{itemize}
%\item [-] \textbf{Cr�ation} : Demand�e par un autre processus,  cr�e dans l'�tat pr�t ou suspendu.
%\item [-] \textbf{Destruction} : peut �tre fait par le processus lui-m�me, un autre processus, le noyau. La destruction provoque une lib�ration des ressources associ�es et un �v�nement.
%\item [-] \textbf{Blocage} : passage en mode bloqu� en attente d'un �v�nement externe. Peut �tre demand� par le processus lui-m�me ou par le syst�me.
%\item [-] \textbf{D�blocage} : passage en mode pr�t apr�s le mode bloqu� lorsque l'�v�nement attendu se produit.
%\item [-] \textbf{Activation} : passage en mode ex�cution d'un mode pr�t.
%\end{itemize}

  
%\begin{Verbatim}[frame=single,fontsize=\scriptsize]
%\end{Verbatim}

%\begin{figure}[h]
%	\begin{center}
%		\includegraphics[width=14.5cm,height=9cm]{.\figures\diag_etat_threads.png}
%	\end{center}
%	\caption{Diagramme des diff�rents �tats d'un thread avec les primitives}
%	\label{fig:diag_etat_threads}
%\end{figure}

%% Modele d'etat avec les matrices

%\begin{displaymath}
%\left\{ \begin{array}{l} \dot{x} \quad = \quad \left[ \begin{array}{cccc}
%0 & 1 & 0 & 1\\
%- \frac{K_s}{M_s} & - \frac{C_s}{M_s} & 0 & \frac{C_s}{M_s}\\
%0 & 0 & 0 & 1\\
%\frac{K_s}{M_u} & \frac{C_s}{M_u} & - \frac{K_t}{M_u} & - \frac{C_s}{M_u} - \frac{C_t}{M_u}
%
%\end{array}\right]  x \quad + \quad \left[ \begin{array}{cc}
%0 & 1.1972\\
%0 & - 0.0012\\
%0 & 0\\
%7.84 & - 4.05
%\end{array} \right] u\\ \\
%y \quad = \quad \left[ \begin{array}{cccc}
%1 & 0 & 0 & 0\\
%0 & \lambda & 0 & 0\\
%0 & 0 & \lambda & 0\\
%0 & 0 & 0 & \lambda\\
%0 & - \lambda & \lambda & 0
%\end{array} \right] x \quad + \quad \left[ \begin{array}{cc}
%0 & 0\\
%0 & 0\\
%0 & 0\\
%0 & 0\\
%0 & 0
%\end{array} \right] u \end{array}\right.
%\end{displaymath}\\

% --------------------------------------------------------------------------------------------------------------------------------------------------------------------------- %
\rhead{\footnotesize\rightmark}
\chapter{Observateur du Kalman}

	\textcolor{red}{Petrov}

	\section{�quations du filtre de Kalman}
	
	\section{Calcul du gain de Kalman}
